Documentación final para la Práctica Primavera 2020. En ella se encuentran tanto los atributos privados como públicos, así como el código final utilizado.

El programa principal de la práctica pertenece al módulo \mbox{\hyperlink{program_8cc}{program.\+cc}}. Este módulo será el encargado de leer la constante k, y de crear un conjunto de especies y un conjunto de clústers vacíos para posteriormente hacer las llamadas a los métodos solicitados y lanzar los mensajes de error en caso de ser necesarios.

Atendiendo a los tipos de datos sugeridos en el enunciado, se necesitan módulos para representar el \mbox{\hyperlink{class_cjt__clusters}{Cjt\+\_\+clusters}} (finalmente se reducirá a un único clúster identificado como árbol filogenético), otro para el \mbox{\hyperlink{class_cjt__especies}{Cjt\+\_\+especies}} y otro para \mbox{\hyperlink{class_especie}{Especie}}.

Entre otras funcionalidades de la práctica, la más importante será la creación de un árbol filogenético de un conjunto de clústers (inicialmente especies), es decir, calcular cuales son los predecesores más cercanos e ir fusionándolos para su posterior impresión.

Lenguaje\+: c++

Versión\+: 3.\+4

Autora\+: Maria Montalvo Falcón

Fecha\+: 19/05/2020 